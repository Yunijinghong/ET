%!TEX program = xelatex
\documentclass[dvipsnames, svgnames,a4paper,11pt]{article}
\input{Settings} 
\usepackage{lipsum}
\usepackage{adjustbox}
\usepackage{indentfirst}
\setlength{\parindent}{2em}
%\usepackage{mathrsfs} % 字体
%\captionsetup[figure]{name=Figure} % 图片形式
%\captionsetup[table]{name=Table} % 表格形式
\begin{document}
	
	% 实验报告封面	
	% 顶栏
	\begin{table}
		\renewcommand\arraystretch{1.7}
		\begin{tabularx}{\textwidth}{
				|X|X|X|X
				|X|X|X|X|}
			\hline
			\multicolumn{2}{|c|}{预习报告}&\multicolumn{2}{|c|}{实验记录}&\multicolumn{2}{|c|}{分析讨论}&\multicolumn{2}{|c|}{总成绩}\\
			\hline
			\LARGE25 & & \LARGE25 & & \LARGE30 & & \LARGE80 & \\
			\hline
		\end{tabularx}
	\end{table}
	% ---
	
	% 信息栏
	\begin{table}
		\renewcommand\arraystretch{1.7}
		\begin{tabularx}{\textwidth}{|X|X|X|X|}
			\hline
			年级、专业: & 2022级 物理学 &组号: &D8 \\
			\hline
			姓名: &  黄罗琳、王显  & 学号: &22344001、22344002   \\
			\hline
			实验时间: & 2024/3/27 & 教师签名: & \\
			\hline
		\end{tabularx}
	\end{table}
	% ---
	
	% 大标题
	\begin{center}
		\LARGE ET5 \quad 一阶电路暂态过程的研究
	\end{center}
	% ---
	
	% 注意事项
	
	% 基本
	\textbf{【实验报告注意事项】}
	\begin{enumerate}
		\item 实验报告由三部分组成:
		\begin{enumerate}
			\item 预习报告:课前认真研读实验讲义,弄清实验原理;实验所需的仪器设备、用具及其使用、完成课前预习思考题;了解实验需要测量的物理量,并根据要求提前准备实验记录表格(可以参考实验报告模板,可以打印)。\textcolor{red}{\textbf{(20分)}}
			\item 实验记录:认真、客观记录实验条件、实验过程中的现象以及数据。实验记录请用珠笔或者钢笔书写并签名(\textcolor{red}{\textbf{用铅笔记录的被认为无效}})。\textcolor{red}{\textbf{保持原始记录,包括写错删除部分,如因误记需要修改记录,必须按规范修改。}}(不得输入电脑打印,但可扫描手记后打印扫描件);离开前请实验教师检查记录并签名。\textcolor{red}{\textbf{(30分)}}
			\item 数据处理及分析讨论:处理实验原始数据(学习仪器使用类型的实验除外),对数据的可靠性和合理性进行分析;按规范呈现数据和结果(图、表),包括数据、图表按顺序编号及其引用;分析物理现象(含回答实验思考题,写出问题思考过程,必要时按规范引用数据);最后得出结论。\textcolor{red}{\textbf{(30分)}}
		\end{enumerate}
		\textbf{实验报告就是将预习报告、实验记录、和数据处理与分析合起来,加上本页封面。\textcolor{red}{(80分)}}
		\item 每次完成实验后的一周内交\textbf{实验报告}(特殊情况不能超过两周)。
		\item \textbf{其它注意事项}:
		\begin{enumerate}
			\item 请认真查看并理解实验讲义第一章内容;
			\item 注意实验器材的合理使用;
			\item 使用结束使用各种仪器之后需要将其放回原位。
		\end{enumerate}
	\end{enumerate}
	
	% 安全
	\textbf{【实验安全注意事项】}	
	\begin{enumerate}
		\item 电路左端为输入端,加方波信号,右端为输出端,接示波器,不要弄错。
		\item 用示波器观察波形,一定要将输入信号选择开关置于“AC”位置,随被测信号幅值不同,改变幅值开关的位置,使波形清晰可测。
		
	\end{enumerate}
	
	% 目录
	\clearpage
	\tableofcontents
	\clearpage
	% ---
	
	
	
	% 预习报告	
	
	% 小标题
	\setcounter{section}{0}
	\section{ET5 一阶电路暂态过程的研究 \quad\heiti 预习报告}
	% ---
	
	% 实验目的
	\subsection{实验目的}
	\begin{enumerate}
		\item 研究一阶电路的零输入响应,零状态响应及全响应的基本规律和特点。
		\item 学习一阶电路时间常数τ的测量方法。
		\item 熟悉微分和积分电路结构,加深对构成微分和积分电路必要条件的理解。
		\item 进一步熟悉应用示波器进行电参数测量的方法。
		
	\end{enumerate}
	% ---
	
	% 仪器用具
	\subsection{仪器用具}
	\begin{table}[htbp]
		\centering
		\renewcommand\arraystretch{1.6}
		% \setlength{\tabcolsep}{10mm}
		\begin{tabular}{p{0.05\textwidth}|p{0.20\textwidth}|p{0.05\textwidth}|p{0.5\textwidth}}
			\hline
			编号& 仪器用具名称 & 数量 &  主要参数(型号,测量范围,测量精度等) \\
			\hline
			1& 电路原理箱或板 & 1 & 一阶电路动态过程的研究 \\
	
			2& 示波器 & 1 &  \\
			
			3& 2号实验导线 & n & 二端2号镀金插头  \\
			\hline

		\end{tabular}
	\end{table}
	% ---
	
	% 原理概述
	\subsection{原理概述}
	
	\begin{enumerate}
		\item \textbf{动态电路及响应}:含有 \( L \)、 \( C \) 元件的电路称为动态电路,其描述方程为微分方程。线性电路的响应可分为零状态响应、零输入响应及全响应,初始状态为零,仅激励引起的响应叫零状态响应;激励为零,由初始条件引起的响应叫零输入响应;同时同激励和初始条件引起的响应叫全响应。
		
		\item \textbf{一阶电路分析}:电路中只含有一个电感或电容元件时称为一阶电路。一阶电路的零输入响应总是按指数规律衰减,零状态响应总是按指数规律递增或递减,衰减和递增速率的快慢,决定于时间常数 $\tau$ 。
		在 RC 电路中,$\tau=RC$;在 RL 电路中, $\tau=\frac LR$。
		
		\item \textbf{过渡过程}:动态电路的过渡过程是短暂的单次变化过程,难以直接观察,所以使用的实验方法即是本实验所使用的:
		 \begin{enumerate}
			\item 使用方波信号作为输入激励。
			\item 用示波器观察电路的输出响应。
			\item 根据电路的时间常数 \( \tau \),调整方波信号的频率和幅度,以便在示波器上观察到清晰的过渡过程。
		\end{enumerate}
		
		\item \textbf{微分电路和积分电路}:微分电路和积分电路是脉冲数字电路中常见的波形变换电路。当电路时间常数 \( \tau \) 远小于或远大于方波脉冲宽度 \( T_p \) 时,微分电路可将方波变换成尖脉冲,积分电路可将方波变换成三角波。电容和电感的数学表达式如下:
		
		\begin{itemize}
			\item 电容: \( i= C \frac{du}{dt} \), \( u = \frac{1}{C} \int i dt + \text{constant} \)
			\item 电感: \( u = L \frac{di}{dt} \), \( i = \frac{1}{L} \int u dt + \text{constant} \)
		\end{itemize}
		
		
		
	\end{enumerate}
	% ---
	% ---
	
	
	
	% 实验前思考题
	\subsection{实验预习题}
	
	% 思考题1
	\begin{question}
		微分电路如图\ref*{fig:graph1}所示,R=5.1KΩ,$C_2$=0.1μF,试问对方波脉宽有什么要求?
	\end{question}



		在微分电路中,方波脉宽的要求与电路的时间常数$\tau$有关。时间常数是由电阻$R$和电容$C$的值决定的,计算公式为$ \tau = R \times C$。对于此电路,R = 5.1KΩ 和 $C_1$​ = 0.1μF,时间常数 $\tau$ 可以计算如下:
		\[
			\tau=R\times C_1​=5.1\times10^3\Omega×0.1×10^{−6}F=5.1×10^{−4}s
		\]
		
		在微分电路中,为了使电路正常工作,方波的脉宽$Tp$应远大于电路的时间常数,通常至少是时间常数的20倍以上。这样,电路才能对输入信号进行有效的微分。如果方波脉宽太短,电容将无法在每个脉冲之间充分放电,导致输出信号失真。
		则对于此电路,方波脉宽的要求大致为:
		\[
			Tp\gg20\tau=20×5.1×10^{−4}s=10.2ms
		\]
		
		这意味着方波脉宽应远大于10.2ms才能确保电路正常工作。

		\begin{figure}[htbp]
			\centering
			\includegraphics[width=0.4\textwidth]{image.png}
			\caption{RC微分电路}
			\label{fig:graph1}
		\end{figure}

	
	
	% ---
	
	
	
	% 实验记录	
	\clearpage
	
	% 顶栏
	\begin{table}
		\renewcommand\arraystretch{1.7}
		\centering
		\begin{tabularx}{\textwidth}{|X|X|X|X|}
			\hline
			专业: & 物理学 & 年级: & 2022级 \\
			\hline
			姓名: & 黄罗琳 、王显& 学号: &22344001、22344002 \\
			\hline
			室温: & 25℃ & 实验地点: & A522 \\
			\hline
			学生签名:&  & 评分: &\\
			\hline
			实验时间:& 2024/3/27 & 教师签名:&\\
			\hline
		\end{tabularx}
	\end{table}
	% ---
	
	% 小标题
	\section{一阶电路暂态过程的研究\quad\heiti 实验记录}
	% ---
	
	% 实验过程记录
	\subsection{实验内容、步骤与结果}
	
	%
	\subsubsection{实验步骤}
\begin{enumerate}
	 \item  RC微分电路如图2所示,调节方波仪输出频率,使方波脉冲宽度满足微分电路的必要条件,将R2、R4分别接入,观察微分电路输出有何不同,并记录相关数据和波形图像。
	 \item RL 微分电路如图3所示,调节方波仪输出频率,使方波脉冲宽度满足微分电路的必要条件,将 R1、 R3分别
	 接入,观察微分电路输出有何不同,并记录相关数据和波形图像。
	 \item RC 积分电路如图4所示,在方波脉冲宽度满足积分电路必要条件下将 R1、 R3 分别接入,观察积分电路输出
	 有何不同, 并记录相关数据和波形图像。
	 \item RL 积分电路如图5所示,在方波脉冲宽度满足积分电路必要条件下,将 R2、 R4 分别接入,观察积分电路输
	 出有何不同, 并记录相关数据和波形图像。
	\end{enumerate}
	 \begin{figure}[H]
		\begin{minipage}[b]{0.5\linewidth}
		  \centering
		  \includegraphics[width=\linewidth]{RC微分电路.png}
		  \caption{RC微分电路(尖脉冲)}
		\end{minipage}
		\hfill
		\begin{minipage}[b]{0.5\linewidth}
		  \centering
		  \includegraphics[width=\linewidth]{RL微分电路.png}
		  \caption{RL微分电路(尖脉冲)}
		\end{minipage}
	 \end{figure}
	 \begin{figure}[H]
		\begin{minipage}[b]{0.5\linewidth}
		  \centering
		  \includegraphics[width=\linewidth]{RC积分电路.png}
		  \caption{RC积分电路}
		\end{minipage}
		\hfill
		\begin{minipage}[b]{0.5\linewidth}
		  \centering
		  \includegraphics[width=\linewidth]{RL积分电路.png}
		  \caption{RL积分电路(三角波)}
		\end{minipage}
	 \end{figure}
	 

	%
	\subsubsection{实验结果}
	\begin{enumerate}
\item RC微分电路
\begin{figure}[{H}]
	\centering
	\includegraphics[width=0.35\linewidth]{RC微分5.1.jpg}
	\caption{RC微分电路5.1KΩ示波器图像}
	\label{}
\end{figure}
实验所设参数为$R_2=5.1\text{k}\Omega\text{,}C_1=0.1\mu\text{F},f=300H\text{z}$\\
\indent 通过光标测得实验数据,此时实验波形图像为尖脉冲,实验参数和结果可以通过图6得知。

\begin{figure}[{H}]
	\centering
	\includegraphics[width=0.35\linewidth]{RC微分2.4.jpg}
	\caption{RC微分电路2.4KΩ示波器图像}
	\label{}
\end{figure}
实验所设参数为$R_4=2.4\text{k}\Omega\text{,}C_1=0.1\mu\text{F},f=300H\text{z}$\\
\indent 通过光标测得实验数据,此时实验波形图像为尖脉冲,实验参数和结果可以通过图7得知。
\item RL微分电路
\begin{figure}[{H}]
	\centering
	\includegraphics[width=0.35\linewidth]{RL微分电路3.0.jpg}
	\caption{RL微分电路3.0KΩ示波器图像}
	\label{}
\end{figure}
实验所设参数为$R_1=3.0\text{k}\Omega\text{,}C_1=0.1\text{H},f=100H\text{z}$\\
\indent 通过光标测得实验数据,此时实验波形图像为尖脉冲,实验参数和结果可以通过图8得知。

\begin{figure}[{H}]
	\centering
	\includegraphics[width=0.35\linewidth]{RL微分5.1.jpg}
	\caption{RL微分电路5.1KΩ示波器图像}
	\label{}
\end{figure}
实验所设参数为$R_3=5.1\text{k}\Omega\text{,}C_1=0.1\text{H},f=100H\text{z}$\\
\indent 通过光标测得实验数据,此时实验波形图像为尖脉冲,实验参数和结果可以通过图9得知。
\item RC积分电路
	\end{enumerate}
	
	% ---
	

	% 问题记录
	\subsection{实验过程遇到问题及解决办法}
	\begin{enumerate}
		\item 
	\end{enumerate}
	% ---
	
	
	
	% 分析与讨论	
	\clearpage
	
	% 顶栏
	\begin{table}
		\renewcommand\arraystretch{1.7}
		\begin{tabularx}{\textwidth}{|X|X|X|X|}
			\hline
			专业:& 物理学 &年级:& 2022级\\
			\hline
			姓名: & 黄罗琳、王显 & 学号:& 22344001、22344002\\
			\hline
			日期:& 2024/3/27 & 评分: &\\
			\hline
		\end{tabularx}
	\end{table}
	% ---
	
	% 小标题
	\section { 一阶电路暂态过程的研究 \quad \heiti 分析与讨论}
	% ---
	
	% 数据处理
	\subsection{实验数据分析与讨论}
	
	%
	\subsubsection{RC微分电路}
	时间常数$\tau=RC$
	\begin{table}[H]
		\centering
		\begin{tabular}{|l|l|l|}
		\hline
			~ & R2=5.1KΩ & R4=2.4KΩ \\ \hline
			理论值 & 510$\mu$s & 240$\mu$s \\ \hline
			测量值 & 508$\mu$s & 250$\mu$s \\ \hline
			相对误差 & -0.39\% & +4.16 \% \\ \hline
		\end{tabular}
	\end{table}
	根据实验数据可知,由于R2的电阻比R4大,故可以时间常数更大一些,符合实验结果,并且实验数据的相对误差较小,无明显问题。\\
	\indent 但是明显看出相对误差出现了变化,这是由于示波器的光标在调整时并不能连续调整,而是以0.2$\mu$s的步长进行调整,如果选用对电压进行调整,则是以0.04V的步长进行调整,这些均会导致实验出现误差,不过均在误差允许范围之内,实验成功。
	
	%
	\subsubsection{RL微分电路}
	时间常数$\tau=\frac LR$
	\begin{table}[H]
		\centering
		\begin{tabular}{|l|l|l|}
		\hline
			~ & R1=3.0KΩ & R3=5.1KΩ \\ \hline
			理论值 & 33.3$\mu$s & 19.6$\mu$s \\ \hline
			测量值 & 32.6$\mu$s & 20.2$\mu$s \\ \hline
			相对误差 & -2.10\% & +3.06\% \\ \hline
		\end{tabular}
	\end{table}
	根据实验数据可知,由于R3的电阻比R1大,所以时间常数更小,符合公式。
	此外实验存在一定的误差,可能来源与RC微分电路实验中误差来源相同,不过均在误差允许范围之内,实验较为成功。
	%
	\subsubsection{}
	
	% ---
	
	% 实验后思考题
	\subsection{实验后思考题}
	
	%思考题1
	\begin{question}
		
	\end{question}
	
	% 思考题2
	\begin{question}
		
	\end{question}
	
	% 思考题3
	\begin{question}
		
	\end{question}
	
	% ---
	
	
	% 结语部分
	\clearpage
	
	% 小标题
	\section{ETX 实验名称××× \quad\heiti 结语}
	% ---
	
	% 总结、杂谈与致谢
	\subsection{实验心得和体会、意见建议等}
	\begin{enumerate}
		\item 
	\end{enumerate}
	% ---
	

	% 附件
	\subsection{附件及实验相关的软硬件资料等}
	试验台桌面整理如%\cref{}所示。
	
	实验报告个人签名如

	% ---
	
	
\end{document}