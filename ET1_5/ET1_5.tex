%!TEX program = xelatex
\documentclass[dvipsnames, svgnames,a4paper,11pt]{article}
\input{Settings} 
\usepackage{lipsum}
\usepackage{adjustbox}
\usepackage{indentfirst}
\setlength{\parindent}{2em}
%\usepackage{mathrsfs} % 字体
%\captionsetup[figure]{name=Figure} % 图片形式
%\captionsetup[table]{name=Table} % 表格形式
\begin{document}
	
	% 实验报告封面	
	% 顶栏
	\begin{table}
		\renewcommand\arraystretch{1.7}
		\begin{tabularx}{\textwidth}{
				|X|X|X|X
				|X|X|X|X|}
			\hline
			\multicolumn{2}{|c|}{预习报告}&\multicolumn{2}{|c|}{实验记录}&\multicolumn{2}{|c|}{分析讨论}&\multicolumn{2}{|c|}{总成绩}\\
			\hline
			\LARGE25 & & \LARGE25 & & \LARGE30 & & \LARGE80 & \\
			\hline
		\end{tabularx}
	\end{table}
	% ---
	
	% 信息栏
	\begin{table}
		\renewcommand\arraystretch{1.7}
		\begin{tabularx}{\textwidth}{|X|X|X|X|}
			\hline
			年级、专业: & 2022级 物理学 &组号: &D8 \\
			\hline
			姓名: &  黄罗琳、王显  & 学号: &22344001、22344002   \\
			\hline
			实验时间: & 2024/3/27 & 教师签名: & \\
			\hline
		\end{tabularx}
	\end{table}
	% ---
	
	% 大标题
	\begin{center}
		\LARGE ET5 \quad 一阶电路暂态过程的研究
	\end{center}
	% ---
	
	% 注意事项
	
	% 基本
	\textbf{【实验报告注意事项】}
	\begin{enumerate}
		\item 实验报告由三部分组成:
		\begin{enumerate}
			\item 预习报告:课前认真研读实验讲义,弄清实验原理;实验所需的仪器设备、用具及其使用、完成课前预习思考题;了解实验需要测量的物理量,并根据要求提前准备实验记录表格(可以参考实验报告模板,可以打印)。\textcolor{red}{\textbf{(20分)}}
			\item 实验记录:认真、客观记录实验条件、实验过程中的现象以及数据。实验记录请用珠笔或者钢笔书写并签名(\textcolor{red}{\textbf{用铅笔记录的被认为无效}})。\textcolor{red}{\textbf{保持原始记录,包括写错删除部分,如因误记需要修改记录,必须按规范修改。}}(不得输入电脑打印,但可扫描手记后打印扫描件);离开前请实验教师检查记录并签名。\textcolor{red}{\textbf{(30分)}}
			\item 数据处理及分析讨论:处理实验原始数据(学习仪器使用类型的实验除外),对数据的可靠性和合理性进行分析;按规范呈现数据和结果(图、表),包括数据、图表按顺序编号及其引用;分析物理现象(含回答实验思考题,写出问题思考过程,必要时按规范引用数据);最后得出结论。\textcolor{red}{\textbf{(30分)}}
		\end{enumerate}
		\textbf{实验报告就是将预习报告、实验记录、和数据处理与分析合起来,加上本页封面。\textcolor{red}{(80分)}}
		\item 每次完成实验后的一周内交\textbf{实验报告}(特殊情况不能超过两周)。
		
	\end{enumerate}
	
	% 安全
	\textbf{【实验安全注意事项】}	
	\begin{enumerate}
		\item 电路左端为输入端,加方波信号,右端为输出端,接示波器,不要弄错。
		\item 用示波器观察波形,一定要将输入信号选择开关置于“AC”位置,随被测信号幅值不同,改变幅值开关的位置,使波形清晰可测。
		
	\end{enumerate}
	
	% 目录
	\clearpage
	\tableofcontents
	\clearpage
	% ---
	
	
	
	% 预习报告	
	
	% 小标题
	\setcounter{section}{0}
	\section{ET5 一阶电路暂态过程的研究 \quad\heiti 预习报告}
	% ---
	
	% 实验目的
	\subsection{实验目的}
	\begin{enumerate}
		\item 研究一阶电路的零输入响应,零状态响应及全响应的基本规律和特点。
		\item 学习一阶电路时间常数τ的测量方法。
		\item 熟悉微分和积分电路结构,加深对构成微分和积分电路必要条件的理解。
		\item 进一步熟悉应用示波器进行电参数测量的方法。
		
	\end{enumerate}
	% ---
	
	% 仪器用具
	\subsection{仪器用具}
	\begin{table}[htbp]
		\centering
		\renewcommand\arraystretch{1.6}
		% \setlength{\tabcolsep}{10mm}
		\begin{tabular}{p{0.05\textwidth}|p{0.20\textwidth}|p{0.05\textwidth}|p{0.5\textwidth}}
			\hline
			编号& 仪器用具名称 & 数量 &  主要参数(型号,测量范围,测量精度等) \\
			\hline
			1& 电路原理箱或板 & 1 & 一阶电路动态过程的研究 \\
	
			2& 示波器 & 1 &  \\
			
			3& 2号实验导线 & n & 二端2号镀金插头  \\
			\hline

		\end{tabular}
	\end{table}
	% ---
	
	% 原理概述
	\subsection{原理概述}
	
	\begin{enumerate}
		\item \textbf{动态电路及响应}:含有 \( L \)、 \( C \) 元件的电路称为动态电路,其描述方程为微分方程。线性电路的响应可分为零状态响应、零输入响应及全响应,初始状态为零,仅激励引起的响应叫零状态响应;激励为零,由初始条件引起的响应叫零输入响应;同时同激励和初始条件引起的响应叫全响应。
		
		\item \textbf{一阶电路分析}:电路中只含有一个电感或电容元件时称为一阶电路。一阶电路的零输入响应总是按指数规律衰减,零状态响应总是按指数规律递增或递减,衰减和递增速率的快慢,决定于时间常数 $\tau$ 。
		在 RC 电路中,$\tau=RC$;在 RL 电路中, $\tau=\frac LR$。
		
		\item \textbf{过渡过程}:动态电路的过渡过程是短暂的单次变化过程,难以直接观察,所以使用的实验方法即是本实验所使用的:
		 \begin{enumerate}
			\item 使用方波信号作为输入激励。
			\item 用示波器观察电路的输出响应。
			\item 根据电路的时间常数 \( \tau \),调整方波信号的频率和幅度,以便在示波器上观察到清晰的过渡过程。
		\end{enumerate}
		
		\item \textbf{微分电路和积分电路}:微分电路和积分电路是脉冲数字电路中常见的波形变换电路。当电路时间常数 \( \tau \) 远小于或远大于方波脉冲宽度 \( T_p \) 时,微分电路可将方波变换成尖脉冲,积分电路可将方波变换成三角波。电容和电感的数学表达式如下:
		
		\begin{itemize}
			\item 电容: \( i= C \frac{du}{dt} \), \( u = \frac{1}{C} \int i dt + \text{constant} \)
			\item 电感: \( u = L \frac{di}{dt} \), \( i = \frac{1}{L} \int u dt + \text{constant} \)
		\end{itemize}
		
		
		
	\end{enumerate}
	% ---
	% ---
	
	
	
	% 实验前思考题
	\subsection{实验预习题}
	
	% 思考题1
	\begin{question}
		微分电路如图\ref*{fig:graph1}所示,R=5.1KΩ,$C_2$=0.1μF,试问对方波脉宽有什么要求?
	\end{question}



		在微分电路中,方波脉宽的要求与电路的时间常数$\tau$有关。时间常数是由电阻$R$和电容$C$的值决定的,公式为$ \tau = R \times C$。对于R = 5.1KΩ 和 $C_1$​ = 0.1μF,时间常数 $\tau$ 可以计算如下:
		\[
			\tau=R\times C_1​=5.1\times10^3\Omega×0.1×10^{−6}F=5.1×10^{−4}s
		\]
		方波脉宽:
		\[
			Tp\gg20\tau=20×5.1×10^{−4}s=10.2ms
		\]
		
		方波脉宽应大于10.2ms才能正常工作。

		\begin{figure}[htbp]
			\centering
			\includegraphics[width=0.4\textwidth]{image.png}
			\caption{RC微分电路}
			\label{fig:graph1}
		\end{figure}

	
	
	% ---
	
	
	
	% 实验记录	
	\clearpage
	
	% 顶栏
	\begin{table}
		\renewcommand\arraystretch{1.7}
		\centering
		\begin{tabularx}{\textwidth}{|X|X|X|X|}
			\hline
			专业: & 物理学 & 年级: & 2022级 \\
			\hline
			姓名: & 黄罗琳 、王显& 学号: &22344001、22344002 \\
			\hline
			室温: & 25℃ & 实验地点: & A522 \\
			\hline
			学生签名:& \includegraphics[width=1cm]{签字.jpg} \includegraphics[width=1cm]{wx.jpg} & 评分: &\\
			\hline
			实验时间:& 2024/3/27 & 教师签名:&\\
			\hline
		\end{tabularx}
	\end{table}
	% ---
	
	% 小标题
	\section{一阶电路暂态过程的研究\quad\heiti 实验记录}
	% ---
	
	% 实验过程记录
	\subsection{实验内容、步骤与结果}
	
	%
	\subsubsection{实验步骤}
\begin{enumerate}
	 \item  RC微分电路如图2所示,调节方波仪输出频率,使方波脉冲宽度满足微分电路的必要条件,将R2、R4分别接入,观察微分电路输出有何不同,并记录相关数据和波形图像。
	 \item RL 微分电路如图3所示,调节方波仪输出频率,使方波脉冲宽度满足微分电路的必要条件,将 R1、 R3分别接入,观察微分电路输出有何不同,并记录相关数据和波形图像。
	 \item RC 积分电路如图4所示,在方波脉冲宽度满足积分电路必要条件下将 R1、 R3 分别接入,得到正确实验图像和在低频率下测得时间常数。
	 \item RL 积分电路如图5所示,在方波脉冲宽度满足积分电路必要条件下,将 R2、 R4 分别接入,记录积分电路的尖脉冲和测得时间常数。
	 \item \textbf{实验相关数据结果见实验数据分析讨论}
	\end{enumerate}
	 \begin{figure}[H]
		\begin{minipage}[b]{0.5\linewidth}
		  \centering
		  \includegraphics[width=\linewidth]{RC微分电路.png}
		  \caption{RC微分电路(尖脉冲)}
		\end{minipage}
		\hfill
		\begin{minipage}[b]{0.5\linewidth}
		  \centering
		  \includegraphics[width=\linewidth]{RL微分电路.png}
		  \caption{RL微分电路(尖脉冲)}
		\end{minipage}
	 \end{figure}
	 \begin{figure}[H]
		\begin{minipage}[b]{0.5\linewidth}
		  \centering
		  \includegraphics[width=\linewidth]{RC积分电路.png}
		  \caption{RC积分电路(三角波)}
		\end{minipage}
		\hfill
		\begin{minipage}[b]{0.5\linewidth}
		  \centering
		  \includegraphics[width=\linewidth]{RL积分电路.png}
		  \caption{RL积分电路(三角波)}
		\end{minipage}
	 \end{figure}
	 

	%
	\subsubsection{实验结果}
	\begin{enumerate}
\item RC微分电路
\begin{figure}[{H}]
	\centering
	\includegraphics[width=0.35\linewidth]{RC微分5.1.jpg}
	\caption{RC微分电路5.1KΩ示波器图像}
	\label{}
\end{figure}
实验所设参数为$R_2=5.1\text{k}\Omega\text{,}\quad C_2=0.1\mu\text{F}, \quad f=300H\text{z}$\\
\indent 通过光标测得实验数据,此时实验波形图像为尖脉冲,实验参数和结果可以通过图6得知。

\begin{figure}[{H}]
	\centering
	\includegraphics[width=0.35\linewidth]{RC微分2.4.jpg}
	\caption{RC微分电路2.4KΩ示波器图像}
	\label{}
\end{figure}
实验所设参数为$R_4=2.4\text{k}\Omega\text{,}\quad C_2=0.1\mu\text{F},\quad f=300H\text{z}$\\
\indent 通过光标测得实验数据,此时实验波形图像为尖脉冲,实验参数和结果可以通过图7得知。
\item RL微分电路
\begin{figure}[{H}]
	\centering
	\includegraphics[width=0.35\linewidth]{RL微分电路3.0.jpg}
	\caption{RL微分电路3.0KΩ示波器图像}
	\label{}
\end{figure}
实验所设参数为$R_1=3.0\text{k}\Omega\text{,}\quad L_1=0.1\text{H},\quad f=100H\text{z}$\\
\indent 通过光标测得实验数据,此时实验波形图像为尖脉冲,实验参数和结果可以通过图8得知。

\begin{figure}[{H}]
	\centering
	\includegraphics[width=0.35\linewidth]{RL微分5.1.jpg}
	\caption{RL微分电路5.1KΩ示波器图像}
	\label{}
\end{figure}
实验所设参数为$R_3=5.1\text{k}\Omega\text{,}\quad L_1=0.1\text{H},\quad f=100H\text{z}$\\
\indent 通过光标测得实验数据,此时实验波形图像为尖脉冲,实验参数和结果可以通过图9得知。
\item RC积分电路
 所得实验图像均为三角波。
\begin{figure}[{H}]
	\centering
	\includegraphics[width=0.3\linewidth]{RC积分电路3-100.jpg}
	\caption{RC积分电路3.0KΩ示波器图像}
	\label{}
\end{figure}
实验所设参数为$R_1=3.0\text{k}\Omega\text{,}\quad C_1=10\mu\text{F},\quad f=100H\text{z}$\\
$V_{pp}=572 \times 10^{-3}V$ \quad $T=10ms$
\begin{figure}[{H}]
	\centering
	\includegraphics[width=0.3\linewidth]{RC积分电路5.1-100.jpg}
	\caption{RC积分电路5.1KΩ示波器图像}
	\label{}
\end{figure}
实验所设参数为$R_3=5.1\text{k}\Omega\text{,}\quad C_1=10\mu\text{F},\quad f=100H\text{z}$\\
$V_{pp}=360 \times 10^{-3}V$ \quad $T=10ms$\\
\textbf{为了测量RC积分电路的时间常数,通过减小频率获取到如下图像}
\begin{figure}[{H}]
	\centering
	\includegraphics[width=0.3\linewidth]{RC积分电路1hz.jpg}
	\caption{RC积分电路低频率3.0KΩ示波器图像}
	\label{}
\end{figure}
实验所设参数为$R_1=3.0\text{k}\Omega\text{,}\quad C_1=10\mu\text{F},\quad f=1H\text{z}$\\
如图12所示,实验结果为测得时间常数$\tau=0.031s$\\
\begin{figure}[{H}]
	\centering
	\includegraphics[width=0.3\linewidth]{RC积分电路5.1 1hz.jpg}
	\caption{RC积分电路低频率5.1KΩ示波器图像}
	\label{}
\end{figure}
连接5.1$k\Omega$电阻,得到图13所示图像
实验所设参数为$R_3=5.1\text{k}\Omega\text{,}\quad C_1=10\mu\text{F},\quad f=1H\text{z}$\\
实验结果为测得时间常数$\tau=0.051s$\\
\item RL积分电路 
 \begin{figure}[{H}]
	\centering
	\includegraphics[width=0.4\linewidth]{RL积分电路5.1.jpg}
	\caption{RL积分电路5.1KΩ示波器图像}
	\label{}
 \end{figure}
 实验所设参数为$R_2=5.1\text{k}\Omega\text{,}\quad L_1=100m\text{H},\quad f=20KH\text{z}$\\
$V_{pp}=2.98 V$ \quad $T=50\mu s$
\begin{figure}[{H}]
	\centering
	\includegraphics[width=0.3\linewidth]{RL积分电路2.4.jpg}
	\caption{RL积分电路2.4KΩ示波器图像}
	\label{}
\end{figure}
实验所设参数为$R_4=2.4\text{k}\Omega\text{,}\quad L_1=100m\text{H},\quad f=20KH\text{z}$\\
$V_{pp}=1.60 V$ \quad $T=50\mu s$\\

\begin{figure}[{H}]
	\centering
	\includegraphics[width=0.3\linewidth]{RL积分1hz5.1.jpg}
	\caption{RL积分电路5.1KΩ低频率示波器图像}
	\label{}
\end{figure}
计算时间常数,将频率调低。\\
实验所设参数为$R_2=5.1\text{k}\Omega\text{,}\quad L_1=100m\text{H},\quad f=100H\text{z}$\\
实验结果为测得时间常数$\tau=1.96 \times 10^{-5}s$


\begin{figure}[{H}]
	\centering
	\includegraphics[width=0.4\linewidth]{RL积分电路1hz2.4.jpg}
	\caption{RL积分电路2.4KΩ低频率示波器图像}
	\label{}
\end{figure}
计算时间常数,将频率调低。\\
实验所设参数为$R_4=2.4\text{k}\Omega\text{,}\quad L_1=100m\text{H},\quad f=100H\text{z}$\\
实验结果为测得时间常数$\tau=4.12 \times 10^{-5}s$


	\end{enumerate}
	
	% ---
	

	% 问题记录
	\subsection{实验过程遇到问题及解决办法}
	\begin{enumerate}
		\item 实验过程中出现了实验仪器操作的问题,但是通过询问老师解决了对于示波器的操作疑问。
		\item 实验中出现了示波器图像出现畸变的情况,通过重新接线,并且改变频率的方式得到了正确的图像。
		
	\end{enumerate}
	% ---
	
	
	
	% 分析与讨论	
	\clearpage
	
	% 顶栏
	\begin{table}
		\renewcommand\arraystretch{1.7}
		\begin{tabularx}{\textwidth}{|X|X|X|X|}
			\hline
			专业:& 物理学 &年级:& 2022级\\
			\hline
			姓名: & 黄罗琳、王显 & 学号:& 22344001、22344002\\
			\hline
			日期:& 2024/3/27 & 评分: &\\
			\hline
		\end{tabularx}
	\end{table}
	% ---
	
	% 小标题
	\section { 一阶电路暂态过程的研究 \quad \heiti 分析与讨论}
	% ---
	
	% 数据处理
	\subsection{实验数据分析与讨论}
	
	%
	\subsubsection{RC微分电路}
	时间常数$\tau=RC$
	\begin{table}[H]
		\centering
		\begin{tabular}{|l|l|l|}
		\hline
			~ & R2=5.1KΩ & R4=2.4KΩ \\ \hline
			理论值 & 510$\mu$s & 240$\mu$s \\ \hline
			测量值 & 508$\mu$s & 250$\mu$s \\ \hline
			相对误差 & -0.39\% & +4.16 \% \\ \hline
		\end{tabular}
	\end{table}
	根据实验数据可知,由于R2的电阻比R4大,故可以时间常数更大一些,符合实验结果,并且实验数据的相对误差较小,无明显问题。\\
	\indent 但是明显看出相对误差出现了变化,这是由于示波器的光标在调整时并不能连续调整,而是以0.2$\mu$s的步长进行调整,如果选用对电压进行调整,则是以0.04V的步长进行调整,这些均会导致实验出现误差,不过均在误差允许范围之内,实验成功。
	
	%
	\subsubsection{RL微分电路}
	时间常数$\tau=\frac LR$
	\begin{table}[H]
		\centering
		\begin{tabular}{|l|l|l|}
		\hline
			~ & R1=3.0KΩ & R3=5.1KΩ \\ \hline
			理论值 & 33.3$\mu$s & 19.6$\mu$s \\ \hline
			测量值 & 32.6$\mu$s & 20.2$\mu$s \\ \hline
			相对误差 & -2.10\% & +3.06\% \\ \hline
		\end{tabular}
	\end{table}
	根据实验数据可知,由于R3的电阻比R1大,所以时间常数更小,符合公式。
	此外实验存在一定的误差,可能来源与RC微分电路实验中误差来源相同,不过均在误差允许范围之内,实验较为成功。
	%
	\subsubsection{RC积分电路}
	实验数据如下表所示
	\begin{table}[H]
		\centering
		
		\begin{tabular}{|c|c|c|}
		\hline
		电阻 & 值&峰峰值 \\
		\hline
		$R_1$ & $3.0\,\text{k}\Omega$&  $572 \times 10^{-3}\,V$\\
		\hline
		$R_3$ & $5.1\,\text{k}\Omega$ &$360 \times 10^{-3}\,V$\\
		
		\hline
		\end{tabular}
		\end{table}
		观察数据可知,在电容C一定时,电阻R越大,则峰峰值越小。\\
        \indent \textbf{时间常数$\tau=RC$}
		\begin{table}[H]
			\centering
			\begin{tabular}{|l|l|l|}
			\hline
				~ & R1=3.0KΩ & R3=5.1KΩ \\ \hline
				理论值 & 0.03s & 0.051s \\ \hline
				测量值 & 0.031s & 0.051s \\ \hline
				相对误差 & +3.3\% & 0\% \\ \hline
			\end{tabular}
		\end{table}
		通过对RC积分电路的时间常数的测量,由于结果较为接近,可能是仪器精密度导致的误差引入。可以认为实验获得成功。



		\subsubsection{RL积分电路}
	实验数据如下表所示
	\begin{table}[H]
		\centering
		
		\begin{tabular}{|c|c|c|}
		\hline
		电阻 & 值&峰峰值 \\
		\hline
		$R_2$ & $5.1\,\text{k}\Omega$&  $2.98V$\\
		\hline
		$R_4$ & $2.4\,\text{k}\Omega$ &$1.60V$\\
		
		\hline
		\end{tabular}
		\end{table}
		观察数据可知,在电感L一定时,电阻R越大,则峰峰值越大。\\
		\indent  \textbf{时间常数$\tau=\frac LR$}
	\begin{table}[H]
		\centering
		\begin{tabular}{|l|l|l|}
		\hline
			~ & R2=5.1KΩ & R4=2.4KΩ \\ \hline
			理论值 & 19.6$\mu$s & 4.176$\mu$s \\ \hline
			测量值 & 19.6$\mu$s & 4.12$\mu$s \\ \hline
			相对误差 & 0.00\% & -1.19\% \\ \hline
		\end{tabular}
	\end{table}
	实验结果相对误差较小,实验较为成功。
	\subsubsection{对于积分电路输出三角波的解释(基于林旭东老师解释)}
	\begin{figure}[{H}]
		\centering
		\includegraphics[width=0.4\linewidth]{lxd.jpg}
		\caption{林老师所绘示意图}
		\label{}
	\end{figure}
	通过林老师的讲解和实验过程中的理解,汇总成如下对于三角波的讨论。\\
	\indent 如果输入信号是方波,那么输出信号就会是三角波。这是因为积分电路对于输入的方波信号进行积分时,会得到一个斜率为正的线性函数。当这个线性函数达到一定的幅度后,方波输出变为零,此时积分电路的输出会达到一个极限值,然后又开始逆向积分,导致输出信号变成负斜率。这样周期性地进行斜率变化,就会产生输出为三角波形式的信号。总的来说,积分电路的输出波形是由输入信号的积分特性所决定的,而输入信号是方波时,输出就会是三角波。\\
	\indent 当方波的频率减小时,对积分电路输出三角波形态会产生显著影响。方波的频率降低意味着每个周期内输入信号的变化率减缓。这会导致积分电路在每个周期内有更多的时间来积分输入信号,因此输出的三角波周期也会相应地延长。然而,频率的降低也可能导致积分电路在每个周期内无法充分积分输入信号。当频率降至某个临界点以下时,积分电路可能无法有效地积分输入信号,从而无法输出连续的三角波形。\\
	\indent \textbf{而在低频率下,所进行的就可以对积分电路时间常数进行测量,这就是实验中RC和RL积分电路的后续时间常数测量实验的原理。}
	
	
	% 结语部分
	\clearpage
	
	% 小标题
	\section{一阶电路暂态过程的研究\quad\heiti 结语}
	% ---
	
	% 总结、杂谈与致谢
	\subsection{实验心得和体会、意见建议等}
	\begin{enumerate}
		\item 实验总体难度不大,并且实验原理比较简单,需要将所学的知识更深层次理解,例如在实验过程中询问了林旭东老师频率对于示波器图像为何会出现三角波的物理内涵。如果不对实验内容有所了解,此实验只会是一个黑盒子实验。
		\item 实验操作过程中锻炼了对于示波器的操作能力,属于实验仪器操作能力得到进步,此外连接实验线路也愈加熟练,实验时间基本上属于缩短趋势。
		\item \textbf{本实验报告采用LATEX编辑,实验分工为黄罗琳同学负责记录数据、编辑报告、数据分析,王显同学负责实验操作、误差分析、数据绘图。}
		
	\end{enumerate}
	 \quad \large \textbf{感谢您对于此篇实验报告的阅读与批改,祝您工作顺利!}
	% ---
	

	% 附件
	\subsection{附件}
	\begin{figure}[H]
		\centering
		\begin{minipage}{0.3\linewidth}
			\centering
			\includegraphics[width=\linewidth]{实验桌面.jpg}
			\caption{实验桌面}
		\end{minipage}
		\hfill
		\begin{minipage}{0.3\linewidth}
			\centering
			\includegraphics[width=\linewidth]{实验数据1.jpg}
			\caption{实验数据1}
		\end{minipage}
		\hfill
		\begin{minipage}{0.3\linewidth}
			\centering
			\includegraphics[width=\linewidth]{实验数据2.jpg}
			\caption{实验数据2及教师签名}
		\end{minipage}
	\end{figure}
	
	
	
	
\end{document}