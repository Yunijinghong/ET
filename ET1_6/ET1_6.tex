%!TEX program = xelatex
\documentclass[dvipsnames, svgnames,a4paper,11pt]{article}
\input{Settings} 
\usepackage{lipsum}
\usepackage{adjustbox}
\usepackage{multirow}
\usepackage{amsmath}

%\usepackage{mathrsfs} % 字体
%\captionsetup[figure]{name=Figure} % 图片形式
%\captionsetup[table]{name=Table} % 表格形式
\begin{document}
	
	% 实验报告封面	
	% 顶栏
	\begin{table}
		\renewcommand\arraystretch{1.7}
		\begin{tabularx}{\textwidth}{
				|X|X|X|X
				|X|X|X|X|}
			\hline
			\multicolumn{2}{|c|}{预习报告}&\multicolumn{2}{|c|}{实验记录}&\multicolumn{2}{|c|}{分析讨论}&\multicolumn{2}{|c|}{总成绩}\\
			\hline
			\LARGE25 & & \LARGE25 & & \LARGE30 & & \LARGE80 & \\
			\hline
		\end{tabularx}
	\end{table}
	% ---
	
	% 信息栏
	\begin{table}
		\renewcommand\arraystretch{1.7}
		\begin{tabularx}{\textwidth}{|X|X|X|X|}
			\hline
			年级、专业: & 2022级 物理学 &组号: &D8 \\
			\hline
			姓名: & 黄罗琳、王显   & 学号: &  22344001、22344002 \\
			\hline
			实验时间: & 2024/4/8 & 教师签名: & \\
			\hline
		\end{tabularx}
	\end{table}
	% ---
	
	% 大标题
	\begin{center}
		\LARGE ET6 R、L、C元件阻抗特性研究 \quad 实验名称
	\end{center}
	% ---
	
	% 注意事项
	
	% 基本
	\textbf{【实验报告注意事项】}
	\begin{enumerate}
		\item 实验报告由三部分组成:
		\begin{enumerate}
			\item 预习报告:课前认真研读实验讲义,弄清实验原理;实验所需的仪器设备、用具及其使用、完成课前预习思考题;了解实验需要测量的物理量,并根据要求提前准备实验记录表格(可以参考实验报告模板,可以打印)。\textcolor{red}{\textbf{(20分)}}
			\item 实验记录:认真、客观记录实验条件、实验过程中的现象以及数据。实验记录请用珠笔或者钢笔书写并签名(\textcolor{red}{\textbf{用铅笔记录的被认为无效}})。\textcolor{red}{\textbf{保持原始记录,包括写错删除部分,如因误记需要修改记录,必须按规范修改。}}(不得输入电脑打印,但可扫描手记后打印扫描件);离开前请实验教师检查记录并签名。\textcolor{red}{\textbf{(30分)}}
			\item 数据处理及分析讨论:处理实验原始数据(学习仪器使用类型的实验除外),对数据的可靠性和合理性进行分析;按规范呈现数据和结果(图、表),包括数据、图表按顺序编号及其引用;分析物理现象(含回答实验思考题,写出问题思考过程,必要时按规范引用数据);最后得出结论。\textcolor{red}{\textbf{(30分)}}
		\end{enumerate}
		\textbf{实验报告就是将预习报告、实验记录、和数据处理与分析合起来,加上本页封面。\textcolor{red}{(80分)}}
		\item 每次完成实验后的一周内交\textbf{实验报告}(特殊情况不能超过两周)。
	\end{enumerate}
		
	
	% 安全
	\textbf{【实验安全注意事项】}	
	\begin{enumerate}
		\item 采用万用表交流电压档测量交流电压时,请注意有效频率范围。
		\item 在使用示波器测量电压或相位差时,注意示波器两个通道输入信号的接地点接法必须保证信号源、示波器所有通道的接地点接在电路的同一个点上。
		\item 在使用示波器测量电压或相位差时,示波器输入信号的耦合方式要选择交流耦合。
		\item 绘制幅频特性和相频特性曲线时,频率轴采用对数坐标。
		\item 在计算时请注意角频率与频率的转换。
	\end{enumerate}
	
	% 目录
	\clearpage
	\tableofcontents
	\clearpage
	% ---
	
	
	
	% 预习报告	
	
	% 小标题
	\setcounter{section}{0}
	\section{ET6 R、L、C元件阻抗特性研究\quad\heiti 预习报告}
	% ---
	
	% 实验目的
	\subsection{实验目的}
	\begin{enumerate}
		\item[1.] 测量电阻、感抗、容抗与频率的关系,测定 $R$-$f$、$X_L$-$f$ 及 $X_C$-$f$ 特性曲线。
		\item[2.] 观察并了解 $R$、$L$、$C$ 元件两端电压与电流间的相位关系。
	\end{enumerate}
	% ---
	
	% 仪器用具
	\subsection{仪器用具}
	\begin{table}[htbp]
		\centering
		\renewcommand\arraystretch{1.6}
		\begin{tabular}{|c|p{3cm}|c|p{5cm}|c|p{2cm}|}
			\hline
			序号 & 名称 & 型号 & 技术特性及说明 & 数量 & 备注 \\
			\hline
			1 & 电路原理箱或板 & & & 1 & \\
			2 & 函数信号发生器 & & & 1 & \\
			3 & 交流毫伏表 & & & 1 & \\
			4 & 直流电压表或万用表 & 0~30V & & 1 & \\
			5 & 2号实验导线 & & 二端2号镀金插头 & N & \\
			\hline
		\end{tabular}
	\end{table}
	

	% ---
	
	% 原理概述
	\subsection{原理概述}
	\subsubsection{元件阻抗表达式}
阻抗 \( Z \) 是电路中元件对交流信号的阻碍程度,对于电阻 \( R \),阻抗是恒定的,为其本身的阻值。对于电感 \( L \) 和电容 \( C \),阻抗与信号频率有关。电感 \( L \) 的阻抗 \( (X_L) \) 与频率成正比,表达式为 \( X_L = 2\pi fL \)。电容 \( C \) 的阻抗 \( (X_C) \) 与频率成反比,表达式为 \( X_C = \frac{1}{2\pi fC} \)。综合电阻 \( R \)、电感 \( L \) 和电容 \( C \) 的阻抗,整个电路的总阻抗 \( Z \) 可以表示为 \( Z = R + j(X_L - X_C) \),其中 \( j \) 为虚数单位。
\begin{table}[htbp]
	\centering
	\caption{R、L、C元件阻抗表达式}
	\begin{tabular}{|c|c|c|c|c|}
		\hline
	  元件 & 阻抗表达式 & 元件 & 阻抗表达式 \\ 
	  \hline
	  电阻 \( R \) & \( R \) & 电感 \( L \) & \( X_L = 2\pi fL \) \\
	  \hline
	  电容 \( C \) & \( X_C = \frac{1}{2\pi fC} \) & 总阻抗 \( Z \) & \( Z = R + j(X_L - X_C) \) \\ 
	  \hline
	\end{tabular}
  \end{table}
\subsubsection{幅频特性与相频特性}
幅频特性曲线描述了电路中电压幅值随频率变化的关系。通过保持输入电压幅值恒定,并改变信号频率,可以绘制出电抗元件在串联电路中电压与频率的关系曲线。相频特性曲线描述了阻抗角(相位差)随频率变化的关系。相位差 \( \phi \) 可以通过双踪示波器测量,根据波形周期和相位差的关系计算得到。

\subsubsection{电压向量三角形与阻抗三角形}
电压向量三角形是指输入电压向量与电抗元件电压向量以及串联电阻电压向量之间的关系。根据三角形相似性原理,这三个电压向量所组成的三角形与整个电路的阻抗三角形相似。电压向量之间的相位关系决定了电压向量三角形的形状和大小,而阻抗三角形则反映了整个电路中各个元件的阻抗相互关系。

通过测量电压和相位差随频率的变化,可以得到电路的幅频特性曲线和相频特性曲线,从而全面了解 \( R \)、 \( L \)、 \( C \) 元件在交流电路中的行为特性。

	% ---
	
	
	
	% 实验前思考题
	\subsection{实验预习题}
	
	% 思考题1
	\begin{question}
		正弦稳态电路中采用向量法简化计算;
	\end{question}
	\begin{itemize}
		\item \textbf{复数表示}:
		  \begin{itemize}
			\item 将正弦波形的电压、电流等参数表示为复数形式。
			\item 例如,一个电压 $V = V_m \sin(\omega t + \phi)$ 可以表示为复数形式 $\hat{V} = V_m e^{j\phi}$,其中 $\hat{V}$ 是电压的复数形式,$V_m$ 是电压幅值,$\phi$ 是相位角,$j$ 是虚数单位。
		  \end{itemize}
		  
		\item \textbf{复数运算}:
		  \begin{itemize}
			\item 在复数形式下,电路中的各种元件(电阻、电感、电容等)都可以用复阻抗来表示。
			\item 例如,电感的复阻抗为 $Z_L = j\omega L$,电容的复阻抗为 $Z_C = \frac{1}{j\omega C}$。
		  \end{itemize}
		  
		\item \textbf{求解电路参数}:
		  \begin{itemize}
			\item 根据电路中各个元件的复阻抗以及复数形式下的欧姆定律和基尔霍夫定律,可以求解电路中各个节点的电压、电流等参数。
			\item 通过相量法的简化计算,可以更快地得到电路的稳态响应,并且能够直观地理解电路中各个参数之间的关系。
		  \end{itemize}
	  \end{itemize}
	% 思考题2
	\begin{question}
		时域和频域表达方式的互相转换.
	\end{question}
	\subsection*{从时域到频域:}
\begin{itemize}
  \item 时域信号函数 $x(t)$ 可以通过傅里叶变换表示为频域的形式 $X(f)$,用以分析信号的频率成分。
  \item 傅里叶变换公式为:
    \[
    X(f) = \int_{-\infty}^{\infty} x(t)e^{-j2\pi ft} dt
    \]
  \item 其中,$X(f)$ 是频率为 $f$ 的信号成分的复幅值,$f$ 是频率,$j$ 是虚数单位。
\end{itemize}

\subsection*{从频域到时域:}
\begin{itemize}
  \item 频域的信号函数 $X(f)$ 可以通过逆傅里叶变换表示为时域的形式 $x(t)$,还原出原始信号。
  \item 逆傅里叶变换公式为:
    \[
    x(t) = \int_{-\infty}^{\infty} X(f)e^{j2\pi ft} df
    \]
  \item 其中,$x(t)$ 是时域信号函数,$X(f)$ 是频域信号的复幅值。
\end{itemize}
	
	% 思考题3
	\begin{question}
		阻抗三角形、电压向量三角形和功率三角形;
	\end{question}
	\subsection*{阻抗三角形:}
	阻抗三角形是指在交流电路中,由电阻、电感和电容组成的三角形,其中各边分别代表电阻、电感和电容的阻抗。根据电路中的阻抗关系,阻抗三角形的任意两边之和等于第三边,符合基尔霍夫定律。
	
	\subsection*{电压向量三角形:}
	电压向量三角形是指在交流电路中,输入电压、电抗元件上的电压和串联电阻上的电压所组成的三角形。根据电路中的电压关系,电压向量三角形的任意两边之和等于第三边,符合基尔霍夫定律。
	
	\subsection*{功率三角形:}
	功率三角形是指在交流电路中,输入功率、有功功率和无功功率所组成的三角形。根据电路中的功率关系,功率三角形的任意两边之和等于第三边,符合功率平衡定律。
	


	\begin{question}
		RC和RL电路的频率特性分析;
	\end{question}
	\subsection*{RC电路的频率特性分析:}
RC电路是由电阻(R)和电容(C)组成的电路。其频率特性分析涉及电压、电流的相位和幅值随频率的变化。
\begin{itemize}
  \item 幅频特性:在RC电路中,随着频率的增加,电容的阻抗减小,导致整体电路的阻抗减小,电压幅值增加。频率越高,电压幅值越大。
  \item 相频特性:RC电路的相位差随频率的变化呈线性变化,相位差随频率增大而增加,最终趋向于90度。
\end{itemize}

\subsection*{RL电路的频率特性分析:}
RL电路是由电阻(R)和电感(L)组成的电路。其频率特性分析同样涉及电压、电流的相位和幅值随频率的变化。
\begin{itemize}
  \item 幅频特性:在RL电路中,随着频率的增加,电感的阻抗增大,导致整体电路的阻抗增大,电压幅值减小。频率越高,电压幅值越小。
  \item 相频特性:RL电路的相位差随频率的变化呈线性变化,相位差随频率增大而减小,最终趋向于0度。
\end{itemize}
	% ---
	
	
	
	% 实验记录	
	\clearpage
	
	% 顶栏
	\begin{table}
		\renewcommand\arraystretch{1.7}
		\centering
		\begin{tabularx}{\textwidth}{|X|X|X|X|}
			\hline
			专业: & 物理学 & 年级: & 2022级 \\
			\hline
			姓名: &黄罗琳、王显  & 学号: &22344001、22344002 \\
			\hline
			室温: &23℃  & 实验地点: & A522 \\
			\hline
			学生签名:&\includegraphics[width=1cm]{签字.jpg} \includegraphics[width=1cm]{wx.jpg}   & 评分: &\\
			\hline
			实验时间:& 2024/4/10 & 教师签名:&\\
			\hline
		\end{tabularx}
	\end{table}
	% ---
	
	% 小标题
	\section{ET6 R、L、C元件阻抗特性研究 \quad\heiti 实验记录}
	% ---
	
	% 实验过程记录
	\subsection{实验内容、步骤与结果}
	
	%
	\subsubsection{测量R、L、C元件的阻抗频率特性}
	\begin{enumerate}
		\item 实验电路图如下:
		
		
		
		
		\item 通过函数信号发生器输出正弦信号接至如图所示电路,作为激励源$U_S$,并用台式万用表交流电压档或示波器测量,激励电压设置为正弦信号输出,无直流偏置,有效值$U_{RMS}=3V$,并在实验过程中保持不变。\\
		使信号源的输出频率从100Hz逐渐增至100KHz左右, 并使端点S分别接通R、L、C三个元件,分别测量UR、Ur,UL、Ur,Uc、Ur,并通过计算得到各频率点时的R、XL与Xc之值。
		\\\textbf{实验数据记录为有效值,示波器通过光标测量为峰峰值,通过计算后得出有效值并记录在表格中。}
		\item 实验数据:
		\begin{table}[h]
			\centering
			\begin{tabular}{|c|c|c|c|c|}
			\hline
			$w(KHZ)$ & $U_r(V)$ & $U_R(V)$ & $R(\Omega)$ & $\phi °$ \\
			\hline
			
			100   & 0.505 & 2.495 & 988.1188119 & $0 $\\
			\hline
			50  & 0.507 & 2.493 & 983.4319527 &$0 $\\
			\hline
			20   & 0.504 & 2.496 & 990.4761905 & $0 $\\
			\hline
			10  & 0.505 & 2.495 & 988.1188119 &$0 $ \\
			\hline
			5   & 0.505 & 2.495 & 988.1188119 &$0 $ \\
			\hline
			2   & 0.505 & 2.495 & 988.1188119 & $0 $\\
			\hline
			1   & 0.505 & 2.495 & 988.1188119 & $0 $\\
			\hline
			0.5   & 0.506 & 2.494 & 985.770751 & $0 $\\
			\hline
			0.2   & 0.504 & 2.496 & 990.4761905 & $0 $\\
			\hline
			0.1   & 0.505 & 2.495 &988.1188119 &$0 $ \\
			\hline
			
			\end{tabular}
			\caption{R元件阻抗实验数据}
			
			\end{table}
			\begin{table}[h]
				\centering
				\begin{tabular}{|c|c|c|c|c|}
				\hline
				$w(KHZ)$ & $U_r(V)$ & $U_L(V)$ & $R(\Omega)$ & $\phi °$ \\
				\hline
				
				100   & 2.772 & 2.495 & 988.1188119 & $0 $\\
				\hline
				50  & 0.507 & 2.493 & 983.4319527 &$0 $\\
				\hline
				20   & 0.504 & 2.496 & 990.4761905 & $0 $\\
				\hline
				10  & 0.505 & 2.495 & 988.1188119 &$0 $ \\
				\hline
				5   & 0.505 & 2.495 & 988.1188119 &$0 $ \\
				\hline
				2   & 0.505 & 2.495 & 988.1188119 & $0 $\\
				\hline
				1   & 0.505 & 2.495 & 988.1188119 & $0 $\\
				\hline
				0.5   & 0.506 & 2.494 & 985.770751 & $0 $\\
				\hline
				0.2   & 0.504 & 2.496 & 990.4761905 & $0 $\\
				\hline
				0.1   & 0.505 & 2.495 &988.1188119 &$0 $ \\
				\hline
				
				\end{tabular}
				\caption{L元件阻抗实验数据}
				
				\end{table}
				\begin{table}[h]
					\centering
					\begin{tabular}{|c|c|c|c|c|}
					\hline
					$w(KHZ)$ & $U_r(V)$ & $U_L(V)$ & $R(\Omega)$ & $\phi °$ \\
					\hline
					
					100   & 2.772 & 2.495 & 988.1188119 & $0 $\\
					\hline
					50  & 0.507 & 2.493 & 983.4319527 &$0 $\\
					\hline
					20   & 0.504 & 2.496 & 990.4761905 & $0 $\\
					\hline
					10  & 0.505 & 2.495 & 988.1188119 &$0 $ \\
					\hline
					5   & 0.505 & 2.495 & 988.1188119 &$0 $ \\
					\hline
					2   & 0.505 & 2.495 & 988.1188119 & $0 $\\
					\hline
					1   & 0.505 & 2.495 & 988.1188119 & $0 $\\
					\hline
					0.5   & 0.506 & 2.494 & 985.770751 & $0 $\\
					\hline
					0.2   & 0.504 & 2.496 & 990.4761905 & $0 $\\
					\hline
					0.1   & 0.505 & 2.495 &988.1188119 &$0 $ \\
					\hline
					
					\end{tabular}
					\caption{C元件阻抗实验数据}
					
					\end{table}
			
	\end{enumerate}	
	
	%
	\subsubsection{}
	\begin{enumerate}
		\item \begin{table}[h]
			\centering
			\caption{表格示例}
			\label{tab:tab1}
			\begin{tabular}{|c|c|c|c|c|c|}
				\hline
				组1/序号i & 1 & 2 & 3 & 4 & 5 \\
				$v_{1i}(m/s)$ & 1.26 & 1.08 & 1.00 & 0.75 & 0.38 \\
				$f_{1i}(Hz)$ & 40073 & 40127 & 40105 & 40088 & 40066 \\
				\hline
				组2/序号i & 1 & 2 & 3 & 4 & 5 \\
				$v_{2i}(m/s)$ & 1.21 & 1.06 & 0.99 & 0.52 & 0.57 \\
				$f_{2i}(Hz)$ & 40143 & 40125 & 40084 & 40080 & 40067 \\
				\hline
				组3/序号i & 1 & 2 & 3 & 4 & 5 \\
				$v_{3i}(m/s)$ & 1.15 & 0.98 & 0.78 & 0.59 & 0.36 \\
				$f_{3i}(Hz)$ & 40135 & 40115 & 40092 & 40070 & 40044 \\
				\hline
			\end{tabular}
		\end{table}		
	\end{enumerate}
	
	% ---
	
	% 原始数据
	\clearpage
	\subsection{原始数据记录}
	实验记录本上的原始数据见%\cref{}(签字)。
	
	实验台桌面整理见%\textbf{附件}部分(\cref{})。
	
	其它原始数据见%\cref{}。
	% ---
	
	% 问题记录
	\subsection{实验过程遇到问题及解决办法}
	\begin{enumerate}
		\item 在实验过程中,起初并没有接入电压表对电源电压进行监测,尽管信号源的输出电压屏幕显示为3V,但是实际上根据一系列不符合理论计算的实验结果说明,信号源的输出电压出现了改变,之后通过接入一个台式万用表对示波器的输出电压进行监测,然后保证不同频率下调整屏幕显示输出电压保证万用表测量电压为3V,得到了正确的实验结果。
	\end{enumerate}
	% ---
	
	
	
	% 分析与讨论	
	\clearpage
	
	% 顶栏
	\begin{table}
		\renewcommand\arraystretch{1.7}
		\begin{tabularx}{\textwidth}{|X|X|X|X|}
			\hline
			专业:& 物理学 &年级:& 2022级\\
			\hline
			姓名: &  & 学号:& \\
			\hline
			日期:&  & 评分: &\\
			\hline
		\end{tabularx}
	\end{table}
	% ---
	
	% 小标题
	\section{ETX 实验名称××× \quad\heiti 分析与讨论}
	% ---
	
	% 数据处理
	\subsection{实验数据分析}
	
	%
	\subsubsection{}
	\begin{enumerate}
		\item 
	\end{enumerate}
	
	%
	\subsubsection{}
	\begin{enumerate}
		\item 
	\end{enumerate}
	
	%
	\subsubsection{}
	
	% ---
	
	% 实验后思考题
	\subsection{实验后思考题}
	
	%思考题1
	\begin{question}
		
	\end{question}
	
	% 思考题2
	\begin{question}
		
	\end{question}
	
	% 思考题3
	\begin{question}
		
	\end{question}
	
	% ---
	
	
	% 结语部分
	\clearpage
	
	% 小标题
	\section{ETX 实验名称××× \quad\heiti 结语}
	% ---
	
	% 总结、杂谈与致谢
	\subsection{实验心得和体会、意见建议等}
	\begin{enumerate}
		\item 
	\end{enumerate}
	% ---
	

	% 附件
	\subsection{附件及实验相关的软硬件资料等}
	试验台桌面整理如%\cref{}所示。
	
	实验报告个人签名如

	% ---
	
	
\end{document}