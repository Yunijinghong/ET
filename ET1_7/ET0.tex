%!TEX program = xelatex
\documentclass[dvipsnames, svgnames,a4paper,11pt]{article}
% ----------------------------------------------------- 
%	加边框的命令
%	参考:https://tex.stackexchange.com/questions/531559/how-to-add-the-page-border-for-first-two-pages-in-latex
\usepackage{tikz}
\usetikzlibrary{calc}
\usepackage{eso-pic}
\AddToShipoutPictureBG{%
\begin{tikzpicture}[overlay,remember picture]
\draw[line width=0.6pt] % 边框粗细
    ($ (current page.north west) + (0.6cm,-0.6cm) $)
    rectangle
    ($ (current page.south east) + (-0.6cm,0.6cm) $); % 边框位置
\end{tikzpicture}}


\usepackage{xcolor}
\definecolor{c1}{HTML}{086173} % 目录颜色 原版为2752C9 紫灰色535AAA 蓝紫色0B0DB7 深蓝色070F94 湖绿色219394 松石灰绿086173
\definecolor{c2}{HTML}{E20129} % 引用颜色 原版\definecolor{c2}{RGB}{190,20,83} 橙色F24729

\usepackage{ctex}
\usepackage[top=28mm,bottom=28mm,left=15mm,right=15mm]{geometry}
\usepackage{hyperref} 
\hypersetup{
	colorlinks,
	linktoc = section, % 超链接位置,选项有section, page, all
	linkcolor = c1, % linkcolor 目录颜色
	citecolor = c1  % citecolor 引用颜色
}
\usepackage{amsmath,enumerate,multirow,float}
\usepackage{tabularx}
\usepackage{tabu}
\usepackage{subfig}
\usepackage{fancyhdr}
\usepackage{graphicx}
\usepackage{wrapfig}  
\usepackage{physics}
\usepackage{appendix}
\usepackage{amsfonts}

%
\usepackage{tcolorbox}
\tcbuselibrary{skins,breakable}
\newtcolorbox{tbox}[2][]{
    colframe=black!70!,
    breakable,
    enhanced,
	boxrule =0.5pt,
    title = {#2},
    fonttitle = \large\kaishu\bfseries,
	drop fuzzy shadow,
    #1
}
\newtcolorbox[auto counter,number within=section]{question}[1][]{
  top=2pt,bottom=2pt,arc=1mm,
  boxrule=0.5pt,
%   frame hidden,
  breakable,
  enhanced, %跨页后不会显示下边框
  coltitle=c1!80!gray,
  colframe=c1,
  colback=c1!3!white,
  drop fuzzy shadow,
  title={思考题~\thetcbcounter:\quad},
  fonttitle=\bfseries,
  attach title to upper,
  #1
}

% ---------------------------------------------------------------------
%	利用cleveref改变引用格式,\cref是引用命令
\usepackage{cleveref}
\crefformat{figure}{#2{\textcolor{c2}{Figure #1}}#3} % 图片的引用格式
\crefformat{equation}{#2{(\textcolor{c2}{#1})}#3} % 公式的引用格式
\crefformat{table}{#2{\textcolor{c2}{Table #1}}#3} % 表格的引用格式


% ---------------------------------------------------------------------
%	页眉页脚设置
\fancypagestyle{plain}{\pagestyle{fancy}}
\pagestyle{fancy}
\lhead{\kaishu 中山大学物理与天文学院电子技术实验\uppercase\expandafter{\romannumeral1}} % 左边页眉,学院 + 课程
\rhead{\kaishu 实验报告By黄罗琳} % 右边页眉,实验报告标题
\cfoot{\thepage} % 页脚,中间添加页码


% ---------------------------------------------------------------------
%	对目录、章节标题的设置
\renewcommand{\contentsname}{\centerline{\huge 目录}}
\usepackage{titlesec}
\usepackage{titletoc}
% \titleformat{章节}[形状]{格式}{标题序号}{序号与标题间距}{标题前命令}[标题后命令]
\titleformat{\section}{\centering\LARGE\songti}{}{1em}{}

% ---------------------------------------------------------------------
%   listing代码环境设置
\usepackage{listings}
\lstloadlanguages{python}
\lstdefinestyle{pythonstyle}{
backgroundcolor=\color{gray!5},
language=python,
frameround=tftt,
frame=shadowbox, 
keepspaces=true,
breaklines,
columns=spaceflexible,                   
basicstyle=\ttfamily\small, % 基本文本设置,字体为teletype,大小为scriptsize
keywordstyle=[1]\color{c1}\bfseries, 
keywordstyle=[2]\color{Red!70!black},   
stringstyle=\color{Purple},       
showstringspaces=false,
commentstyle=\ttfamily\scriptsize\color{green!40!black},%注释文本设置,字体为sf,大小为smaller
tabsize=2,
morekeywords={as},
morekeywords=[2]{np, plt, sp},
numbers=left, % 代码行数
numberstyle=\it\tiny\color{gray}, % 代码行数的数字字体设置
stepnumber=1,
rulesepcolor=\color{gray!30!white}
}




% ---------------------------------------------------------------------
%	其他设置
\def\degree{${}^{\circ}$} % 角度
\graphicspath{{./images/}} % 插入图片的相对路径
\allowdisplaybreaks[4]  %允许公式跨页 
\usepackage{lipsum}
\usepackage{adjustbox}
\usepackage{rotating}
%\usepackage{mathrsfs} % 字体
%\captionsetup[figure]{name=Figure} % 图片形式
%\captionsetup[table]{name=Table} % 表格形式
\begin{document}
	
	% 实验报告封面	
	% 顶栏

	
	% 信息栏
	
	
	% 目录

	

% 设置页面尺寸

\begin{sidewaystable}[!ht]
    \centering
    \resizebox{\textheight}{!}{%
        \begin{tabular}{|*{10}{p{2cm}|}}
            \hline
            1N4007 & ~ & ~ & ~ & ~ & ~ & ~ & ~ & ~ & ~ \\[0.35cm] \hline
            电压U/V & ~ & ~ & ~ & ~ & ~ & ~ & ~ & ~ & ~ \\[0.35cm] \hline
            电流I/mA & ~ & ~ & ~ & ~ & ~ & ~ & ~ & ~ & ~ \\[0.35cm] \hline
            电压U/V & ~ & ~ & ~ & ~ & ~ & ~ & ~ & ~ & ~ \\[0.35cm] \hline
            电流I/mA & ~ & ~ & ~ & ~ & ~ & ~ & ~ & ~ & ~ \\[0.35cm] \hline
            2CW54 & ~ & ~ & ~ & ~ & ~ & ~ & ~ & ~ & ~ \\[0.35cm] \hline
            电压U/V & ~ & ~ & ~ & ~ & ~ & ~ & ~ & ~ & ~ \\[0.35cm] \hline
            电流I/mA & ~ & ~ & ~ & ~ & ~ & ~ & ~ & ~ & ~ \\[0.35cm] \hline
            电压U/V & ~ & ~ & ~ & ~ & ~ & ~ & ~ & ~ & ~ \\[0.35cm] \hline
            电流I/mA & ~ & ~ & ~ & ~ & ~ & ~ & ~ & ~ & ~ \\[0.35cm] \hline
            二极管蓝光 & ~& ~ & ~ & ~ & ~ & ~ & ~ & ~ & ~ \\[0.35cm] \hline
            电压U/V & ~ & ~ & ~ & ~ & ~ & ~ & ~ & ~ & ~ \\[0.35cm] \hline
            电流I/mA & ~ & ~ & ~ & ~ & ~ & ~ & ~ & ~ & ~ \\[0.35cm] \hline
            电压U/V & ~ & ~ & ~ & ~ & ~ & ~ & ~ & ~ & ~ \\[0.35cm] \hline
            电流I/mA & ~ & ~ & ~ & ~ & ~ & ~ & ~ & ~ & ~ \\[0.35cm] \hline
            二极管红光 & ~ & ~ & ~ & ~ & ~ & ~ & ~ & ~ & ~ \\[0.35cm] \hline
            电压U/V & ~ & ~ & ~ & ~ & ~ & ~ & ~ & ~ & ~ \\[0.35cm] \hline
            电流I/mA & ~ & ~ & ~ & ~ & ~ & ~ & ~ & ~ & ~ \\[0.35cm] \hline
            电压U/V & ~ & ~ & ~ & ~ & ~ & ~ & ~ & ~ & ~ \\[0.35cm] \hline
            电流I/mA & ~ & ~ & ~ & ~ & ~ & ~ & ~ & ~ & ~ \\[0.35cm] \hline
        \end{tabular}%
    }
\end{sidewaystable}
\begin{sidewaystable}[!ht]
    \centering
    \begin{tabular}{|*{10}{p{2cm}|}}
        \hline
        NPN型三极管3DG6 & ~ & ~ & ~ & ~ & ~ & ~ & ~ & ~ & ~ \\[0.35cm] \hline
        VCC/V & ~ & ~ & ~ & ~ & ~ & ~ & ~ & ~ & ~ \\[0.35cm] \hline
        VCE/V & ~ & ~ & ~ & ~ & ~ & ~ & ~ & ~ & ~ \\[0.35cm] \hline
        i/mA & ~ & ~ & ~ & ~ & ~ & ~ & ~ & ~ & ~ \\[0.35cm] \hline
        VCC/V & ~ & ~ & ~ & ~ & ~ & ~ & ~ & ~ & ~ \\[0.35cm] \hline
        VCE/V & ~ & ~ & ~ & ~ & ~ & ~ & ~ & ~ & ~ \\[0.35cm] \hline
        i/mA & ~ & ~ & ~ & ~ & ~ & ~ & ~ & ~ & ~ \\[0.35cm] \hline
        VCC/V & ~ & ~ & ~ & ~ & ~ & ~ & ~ & ~ & ~ \\[0.35cm] \hline
        VCE/V & ~ & ~ & ~ & ~ & ~ & ~ & ~ & ~ & ~ \\[0.35cm] \hline
        i/mA & ~ & ~ & ~ & ~ & ~ & ~ & ~ & ~ & ~ \\[0.35cm] \hline
    \end{tabular}
\end{sidewaystable}

\begin{sidewaystable}[!ht]
    \centering
    \begin{tabular}{|*{10}{p{2cm}|}}
        \hline
        NPN型三极管3DG6 & ~ & ~ & ~ & ~ & ~ & ~ & ~ & ~ & ~ \\[0.35cm] \hline
        VCC/V & ~ & ~ & ~ & ~ & ~ & ~ & ~ & ~ & ~ \\[0.35cm] \hline
        VCE/V & ~ & ~ & ~ & ~ & ~ & ~ & ~ & ~ & ~ \\[0.35cm] \hline
        i/mA & ~ & ~ & ~ & ~ & ~ & ~ & ~ & ~ & ~ \\[0.35cm] \hline
        VCC/V & ~ & ~ & ~ & ~ & ~ & ~ & ~ & ~ & ~ \\[0.35cm] \hline
        VCE/V & ~ & ~ & ~ & ~ & ~ & ~ & ~ & ~ & ~ \\[0.35cm] \hline
        i/mA & ~ & ~ & ~ & ~ & ~ & ~ & ~ & ~ & ~ \\[0.35cm] \hline
        VCC/V & ~ & ~ & ~ & ~ & ~ & ~ & ~ & ~ & ~ \\[0.35cm] \hline
        VCE/V & ~ & ~ & ~ & ~ & ~ & ~ & ~ & ~ & ~ \\[0.35cm] \hline
        i/mA & ~ & ~ & ~ & ~ & ~ & ~ & ~ & ~ & ~ \\[0.35cm] \hline
		Vbb/V & ~ & ~ & ~ & ~ & ~ & ~ & ~ & ~ & ~ \\[0.35cm] \hline
        VBE/V & ~ & ~ & ~ & ~ & ~ & ~ & ~ & ~ & ~ \\[0.35cm] \hline
        i/mA & ~ & ~ & ~ & ~ & ~ & ~ & ~ & ~ & ~ \\[0.35cm] \hline
        Vbb/V & ~ & ~ & ~ & ~ & ~ & ~ & ~ & ~ & ~ \\[0.35cm] \hline
        VBE/V & ~ & ~ & ~ & ~ & ~ & ~ & ~ & ~ & ~ \\[0.35cm] \hline
        i/mA & ~ & ~ & ~ & ~ & ~ & ~ & ~ & ~ & ~ \\[0.35cm] \hline
        Vbb/V & ~ & ~ & ~ & ~ & ~ & ~ & ~ & ~ & ~ \\[0.35cm] \hline
        VCBE/V & ~ & ~ & ~ & ~ & ~ & ~ & ~ & ~ & ~ \\[0.35cm] \hline
        i/mA & ~ & ~ & ~ & ~ & ~ & ~ & ~ & ~ & ~ \\[0.35cm] \hline
    \end{tabular}
\end{sidewaystable}

	
	\clearpage
	% 预习报告	
	
	% 小标题
	
	% ---
	\begin{table}[h!]
		\centering
		\begin{tabular}{|c|c|c|c|}
		\hline
		\textbf{实验装置} & \textbf{装置一} & \textbf{使用电压/电流} & \textbf{稳压二极管CW54} \\
		\hline
		毫安表IN4007 & 伏特表量程=100Ω & 电压/V ...... & 电流/mA ...... \\
		\hline
		稳定二极管(短头) & 伏特表量程=100Ω & 电压/V ...... & 电流/mA ...... \\
		\hline
		稳定二极管(长头) & 伏特表量程=100Ω & Vce<\( \frac{1}{4} \)Ubb ...... &  \\
		\hline
		NPN型三极管3DG6(3CG12) & 测定输入、输出特性曲线 & Vbb= ...... & Rb= ...... \\
		\hline
		NPN型三极管3DG6(3CG12) & 测定Vce= & Rc= ...... & 改变Vbb和Rb的值 ...... \\
		\hline
		\end{tabular}
		\caption{实验装置参数表}
		\label{table:experiment_device_parameters}
		\end{table}
		
	% 仪器用具
	\subsection{仪器用具}
	\begin{table}[htbp]
		\centering
		\renewcommand\arraystretch{1.6}
		% \setlength{\tabcolsep}{10mm}
		\begin{tabular}{p{0.05\textwidth}|p{0.20\textwidth}|p{0.05\textwidth}|p{0.5\textwidth}}
			\hline
			编号& 仪器用具名称 & 数量 &  主要参数(型号,测量范围,测量精度等) \\
			\hline
			1&  & 1 &  \\
			\hline
		\end{tabular}
	\end{table}
	% ---
	
	% 原理概述
	\subsection{原理概述}
	\begin{enumerate}
		\item 
	\end{enumerate}
	% ---
	
	
	
	% 实验前思考题
	\subsection{实验预习题}
	
	% 思考题1
	\begin{question}
		
	\end{question}
	
	% 思考题2
	\begin{question}
		
	\end{question}
	
	% 思考题3
	\begin{question}
		
	\end{question}
	
	% ---
	
	
	
	% 实验记录	
	\clearpage
	
	% 顶栏
	\begin{table}
		\renewcommand\arraystretch{1.7}
		\centering
		\begin{tabularx}{\textwidth}{|X|X|X|X|}
			\hline
			专业: & 物理学 & 年级: & 2022级 \\
			\hline
			姓名: &  & 学号: & \\
			\hline
			室温: &  & 实验地点: & A522 \\
			\hline
			学生签名:& 见\textbf{附件}部分 & 评分: &\\
			\hline
			实验时间:& 2024// & 教师签名:&\\
			\hline
		\end{tabularx}
	\end{table}
	% ---
	
	% 小标题
	\section{ETX 实验名称×××  \quad\heiti 实验记录}
	% ---
	
	% 实验过程记录
	\subsection{实验内容、步骤与结果}
	
	%
	\subsubsection{操作步骤记录}
	\begin{enumerate}
		\item 
	\end{enumerate}	
	
	%
	\subsubsection{}
	\begin{enumerate}
		\item \begin{table}[h]
			\centering
			\caption{表格示例}
			\label{tab:tab1}
			\begin{tabular}{|c|c|c|c|c|c|}
				\hline
				组1/序号i & 1 & 2 & 3 & 4 & 5 \\
				$v_{1i}(m/s)$ & 1.26 & 1.08 & 1.00 & 0.75 & 0.38 \\
				$f_{1i}(Hz)$ & 40073 & 40127 & 40105 & 40088 & 40066 \\
				\hline
				组2/序号i & 1 & 2 & 3 & 4 & 5 \\
				$v_{2i}(m/s)$ & 1.21 & 1.06 & 0.99 & 0.52 & 0.57 \\
				$f_{2i}(Hz)$ & 40143 & 40125 & 40084 & 40080 & 40067 \\
				\hline
				组3/序号i & 1 & 2 & 3 & 4 & 5 \\
				$v_{3i}(m/s)$ & 1.15 & 0.98 & 0.78 & 0.59 & 0.36 \\
				$f_{3i}(Hz)$ & 40135 & 40115 & 40092 & 40070 & 40044 \\
				\hline
			\end{tabular}
		\end{table}		
	\end{enumerate}
	
	% ---
	
	% 原始数据
	\clearpage
	\subsection{原始数据记录}
	实验记录本上的原始数据见%\cref{}(签字)。
	
	实验台桌面整理见%\textbf{附件}部分(\cref{})。
	
	其它原始数据见%\cref{}。
	% ---
	
	% 问题记录
	\subsection{实验过程遇到问题及解决办法}
	\begin{enumerate}
		\item 
	\end{enumerate}
	% ---
	
	
	
	% 分析与讨论	
	\clearpage
	
	% 顶栏
	\begin{table}
		\renewcommand\arraystretch{1.7}
		\begin{tabularx}{\textwidth}{|X|X|X|X|}
			\hline
			专业:& 物理学 &年级:& 2022级\\
			\hline
			姓名: &  & 学号:& \\
			\hline
			日期:&  & 评分: &\\
			\hline
		\end{tabularx}
	\end{table}
	% ---
	
	% 小标题
	\section{ETX 实验名称××× \quad\heiti 分析与讨论}
	% ---
	
	% 数据处理
	\subsection{实验数据分析}
	
	%
	\subsubsection{}
	\begin{enumerate}
		\item 
	\end{enumerate}
	
	%
	\subsubsection{}
	\begin{enumerate}
		\item 
	\end{enumerate}
	
	%
	\subsubsection{}
	
	% ---
	
	% 实验后思考题
	\subsection{实验后思考题}
	
	%思考题1
	\begin{question}
		
	\end{question}
	
	% 思考题2
	\begin{question}
		
	\end{question}
	
	% 思考题3
	\begin{question}
		
	\end{question}
	
	% ---
	
	
	% 结语部分
	\clearpage
	
	% 小标题
	\section{ETX 实验名称××× \quad\heiti 结语}
	% ---
	
	% 总结、杂谈与致谢
	\subsection{实验心得和体会、意见建议等}
	\begin{enumerate}
		\item 
	\end{enumerate}
	% ---
	

	% 附件
	\subsection{附件及实验相关的软硬件资料等}
	试验台桌面整理如%\cref{}所示。
	
	实验报告个人签名如

	% ---
	
	
\end{document}